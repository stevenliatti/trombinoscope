\documentclass{article}
\usepackage[T1]{fontenc}
\usepackage[french]{babel}
\usepackage[a4paper, top=2.5cm, right=1.5cm, bottom=2.5cm, left=1.5cm]{geometry}
\usepackage{graphicx}
\usepackage{subcaption}
\usepackage{hyperref}
\hypersetup{
    colorlinks,
    citecolor=black,
    filecolor=black,
    linkcolor=black,
    urlcolor=blue
}

\graphicspath{ {./thumbs/} }

\begin{document}

\title{Trombinoscope du choeur de l'Université de Genève}
\author{Le comité du choeur}
\date{\today}
\maketitle

\begin{figure}[h!]
    \begin{center}
        \includegraphics[width=0.7\textwidth]{in/staff1.jpg}
        \\
        @STAFF_1
    \end{center}
\end{figure}
\begin{figure}[h!]
    \begin{center}
        \includegraphics[width=0.35\textwidth]{in/staff2.jpg}
        \\
        @STAFF_2
    \end{center}
\end{figure}

\bigbreak

\begin{center}
    Trombinoscope généré avec \href{https://www.scala-lang.org/}{Scala}, \href{https://imagemagick.org/}{ImageMagick} et \LaTeX.
    \bigbreak
    Code source disponible sous \href{https://www.gnu.org/licenses/gpl-3.0.fr.html}{GPLv3} sur \url{https://github.com/stevenliatti/trombinoscope}.
\end{center}

\newpage

\input{out/data.tex}

\newpage

\end{document}