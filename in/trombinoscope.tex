\documentclass{article}
\usepackage[T1]{fontenc}
\usepackage[french]{babel}
\usepackage[a4paper, top=1.5cm, right=1.5cm, bottom=1.5cm, left=1.5cm]{geometry}
\usepackage{graphicx}
\usepackage{subcaption}
\usepackage{hyperref}
\hypersetup{
    colorlinks,
    citecolor=black,
    filecolor=black,
    linkcolor=black,
    urlcolor=blue
}

\graphicspath{ {./thumbs/} }

\begin{document}

\title{Trombinoscope du choeur de l'Université de Genève}
\author{Le comité du choeur}
\date{\today}
\maketitle

\begin{figure}[h!]
    \begin{center}
        \includegraphics[width=0.9\textwidth]{in/director.jpg}
        @DIRECTOR - Directeur du choeur
    \end{center}
    % \caption{légende}
\end{figure}

\newpage

\input{out/data.tex}

\newpage

\pagebreak
\hspace{0pt}
\vfill
\begin{center}
    \Large{Généré avec \href{https://www.scala-lang.org/}{Scala}, \href{https://imagemagick.org/}{ImageMagick} et \LaTeX.}
    \bigbreak
    \Large{Code source disponible sous \href{https://www.gnu.org/licenses/gpl-3.0.fr.html}{GPLv3} sur \url{https://github.com/stevenliatti/trombinoscope}.}
\end{center}
\vfill
\hspace{0pt}
\pagebreak

\end{document}